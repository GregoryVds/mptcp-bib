This section documents the different Multipath TCP implementations \cite{draft-eardley-mptcp-implementations-survey} and several interesting applications that have been designed with Multipath TCP in mind.

\subsection{Multipath TCP implementations}

The first implementation of Multipath TCP written by Costin Raiciu inside a userspace port of the Linux TCP stack\footnote{Available from \url{http://nrg.cs.ucl.ac.uk/mptcp/implementation.html}.}. This implementation has been used to perform some initial validations of the protocol and the LIA congestion control scheme \cite{Wischik_Design:2011}, but it was not useable by real applications.

The first in-kernel implementation of Multipath TCP was written by Sebastien Barre in the Linux kernel with the help of several other researchers. It is described in more details in \cite{Barre_thesis:2011}. Some information about this early implementation may be found in \cite{draft-barre-mptcp-impl} but this document has not been updated. This initial implementation supported the LIA congestion control scheme but did not support TSO/GRO nor the mechanisms that are required to deal with middlebox 
interference.

Christoph Paasch with the help of Gregory Detal continued the development of the Multipath TCP patch in the Linux kernel. They ported the implementation over the 3.x kernel releases and added support for TSO/GRO and middelbox traversal techniques. This implementation is described in \cite{Paasch_Thesis:2014}. Several parts of this implementation have been described in more details.

In \cite{Paasch_schedulers:2014}, Paasch et al. propose and implement more flexible packet schedulers for Multipath TCP. In the reference implementation of Multipath TCP in the Linux kernel, the default scheduler selects the subflow having the lowest round-trip-time. In addition to this scheduler, \cite{Paasch_schedulers:2014} proposes a round-robin scheduler that takes to congestion windows of the available subflows into account.  

The next step for the Multipath TCP in the Linux kernel is to be accepted in the mainstream Linux kernel tree. Gucea and Purdila describe in \cite{Gucea_Shaping:2015} the current status of this implementation and the changes that will be required to be accepted upstream.

The Multipath TCP implementation in the Linux kernel is open-source and available from \url{https://github.com/multipath-tcp/mptcp}.

The most widely deployed implementation of Multipath TCP is developed by Apple and is included in both iOS and MacOS. This implementation is open-source. The latest public version is available from \url{http://www.opensource.apple.com/source/xnu/xnu-2782.1.97/bsd/netinet/}. Unfortunately, besides \cite{draft-eardley-mptcp-implementations-survey}, there are no public details on the performance of this implementation and as of this writing, there is no public API that enables an application to use Multipath TCP to send data on an Apple device.

Citrix has also developed an implementation of Multipath TCP for the Netscaler load-balancer. The only published details of this implementation are reported in~\cite{draft-eardley-mptcp-implementations-survey}.

Nigel Williams and his colleagues develop an implementation of Multipath TCP for FreeBSD 11. This implementation is described in details in~\cite{Williams_Design:2014}. It is available from \url{http://caia.swin.edu.au/urp/newtcp/mptcp/}

\subsection{Test tools}
%======================

Several software tools have been designed to aid Multipath TCP implementors and researchers. 

Coene proposed in \cite{draft-coene-mptcp-conformance} to validate a Multipath TCP implementation. These tests have not been maintained, but Schils and Creciun have extended \texttt{packetdrill} \cite{Cardwell_packetdrill:2013} to support Multipath TCP. This extension is available from \url{https://github.com/aschils/packetdrill_mptcp}. It can be used to develop unit tests to validate some specific behaviours of a Multipath TCP stack.

\texttt{tracebox} is a traceroute-like application that was initially designed by Gregory Deal \cite{Detal_tracebox:2013}. \texttt{tracebox} includes a small language that enables researchers to write specific tests to expose middleboxes or interact with a Multipath TCP stack. \texttt{tracebox} is available from \url{http://www.tracebox.org}. Another way to inject Multipath TCP packets is to use the \texttt{scapy} extension\footnote{See \url{https://github.com/nimai/mptcp-scapy}.} developed by Nicolas Maitre that has been recently extended and is available from \url{https://github.com/Neohapsis/mptcp-abuse}.



TCP researchers often rely on \texttt{tcptrace}\footnote{\url{http://tcptrace.org}} to graphically analyse packet traces from TCP connections. Hesmans designed and implemented a similar tool \cite{Hesmans_Tracing:2014} that allows to graphically analyse Multipath TCP connections.  



\subsection{Simulators}
%=======================

Several simulators have been used to evaluate the performance of Multipath TCP and design congestion control schemes or protocol extensions. \texttt{htsim}~\footnote{Available from \url{http://nrg.cs.ucl.ac.uk/mptcp/implementation.html}.} is the original simulator that was used notably for \cite{Wischik_Design:2011,Raiciu_Datacenter:2011}.

Several researchers have used \texttt{ns-2} to evaluate the performance of Multipath TCP starting from the initial model developed by Yoshifumi Nishida \cite{mptcp_ns2}. Two Multipath TCP models have been developed for \texttt{ns-3}. The first one was proposed by Chihani and Collange in \cite{Chihani_mptcp:2011}. 

Another ns-3 model was proposed recently \cite{Kheirkhah_mptcp:2014}. With ns-3, it is also possible to use the complete Multipath TCP implementation in the Linux kernel by leveraging the ns-3-dce extension \cite{Tazaki_dce:2013}.

Besides simulators, it is also possible to perform reproducible experiments with Multipath TCP by leveraging the namespaces that are supported by recent Linux kernels \cite{Handigol_mininet:2012}. Several researchers have relied on this framework to evaluate modifications to the Multipath TCP implementation \cite{Paasch_schedulers:2014,Paasch_Experimental:2013}. 




